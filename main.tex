%--------------------------------------------------------------------
%--------------------------------------------------------------------
%
% Bj\"oern Hartmann
% NSF CAREER 2010 Proposal 
% Copied from James O'Brien's MICRO Proposal, Creative IT 09 proposal
%
%--------------------------------------------------------------------

\ProvidesFile{bio-hartmann-ccf-medium2017.tex}
%\documentclass[10pt,onecolumn,letterpaper]{article}
\documentclass[11pt]{article}

\usepackage{times}
%\usepackage{palatino}
%\usepackage{arial}
\usepackage{graphicx}
\usepackage{wrapfig}
\usepackage{fancyhdr}
\usepackage{cite}
\usepackage{ifthen}
\usepackage[usenames]{color}
\usepackage{url}

%--------------------------------------------------------------------

%\flushbottom

\setlength{\parindent     }{   0in}
\setlength{\parskip       }{.075in}

\setlength{\oddsidemargin }{0in}
\setlength{\evensidemargin}{0in}
\setlength{\topmargin     }{0in}
%\setlength{\headsep       }{  .15in}
\setlength{\headsep       }{  .2in}
\setlength{\textheight    }{ 8.5in}
\setlength{\textwidth     }{ 6.3in}
\setlength{\parindent     }{   0in}
\setlength{\headheight    }{   0in}
\setlength{\parindent     }{   0in}

%--------------------------------------------------------------------
% Creating the pcode command for writing psuedocode
\newcommand{\pcline }{\rule{0in}{0.0in}    }  % Spacing for pseudo-code.
\newcommand{\pctab  }{\hspace{0.10in}      }  % Pseudo-code indentation.
\newcommand{\pcbigtab  }{\hspace{0.1in}    }  % Pseudo-code indentation.
\newcommand{\pcasgn }{\mbox{$\leftarrow$}  }  % Pseudo-code assignment operato
\newcommand{\pcomment}[1]{{\it #1}}  % Pseudo-code comments.

\newcommand{\pcode}[1]{
    \vspace{-0.1in}
    \begin{minipage}{100in} % The width argument will be ignored:
                            % See lamport p. 99
    \begin{tabbing} \hspace{0.20in} \= \pcbigtab \= \pcbigtab \= \pcbigtab \=
\pcbigtab \= \pcbigtab \= \\
       #1
    \end{tabbing}
    \end{minipage}
%    \vspace{0.15in}
}
%--------------------------------------------------------------------

\newcommand{\assA}{-1ex}
\newcommand{\assB}{-2ex}

\newcommand{\assAs}{-1ex}
\newcommand{\assBs}{-2ex}

\newcommand{\assAss}{-1ex}
\newcommand{\assBss}{-2ex}

\newcommand{\assAp}{-1ex}
\newcommand{\assBp}{-2ex}

\newcommand{\assAsp}{-1ex}
\newcommand{\assBsp}{-2ex}

\let\ORIGsection\section
\newcommand{\sectionname}{Section Name}
\def\section#1{\ifthenelse{\equal{#1}{*}}%
	{\sectionS}%
	{\renewcommand{\sectionname}{#1}\vspace{\assA}\ORIGsection{#1}\vspace{\assB}}}%
\def\sectionS#1{\renewcommand{\sectionname}{#1}\vspace{\assA}\ORIGsection*{#1}\vspace{\assB}}%

\let\ORIGsubsection\subsection
\def\subsection#1{\ifthenelse{\equal{#1}{*}}%
	{\subsectionS}%
	{\vspace{\assAs}\ORIGsubsection{#1}\vspace{\assBs}}}%
\def\subsectionS#1{\vspace{\assAs}\ORIGsubsection*{#1}\vspace{\assBs}}%


\let\ORIGsubsubsection\subsubsection
\def\subsubsection#1{\ifthenelse{\equal{#1}{*}}%
	{\subsubsectionS}%
	{\vspace{\assAss}\ORIGsubsubsection{#1}\vspace{\assBss}}}%
\def\subsubsectionS#1{\vspace{\assAss}\ORIGsubsubsection*{#1}\vspace{\assBss}}%


\let\ORIGparagraph\paragraph
\def\paragraph#1{\ifthenelse{\equal{#1}{*}}%
	{\paragraphS}%
	{\vspace{\assAp}\ORIGparagraph{#1}}}%
\def\paragraphS#1{\vspace{\assAp}\ORIGparagraph*{#1}}%


\let\ORIGsubparagraph\subparagraph
\def\subparagraph#1{\ifthenelse{\equal{#1}{*}}%
	{\subparagraphS}%
	{\vspace{\assAsp}\ORIGsubparagraph{#1}}}%
\def\subparagraphS#1{\vspace{\assAsp}\ORIGsubparagraph*{#1}}%

%--------------------------------------------------------------------
%--------------------------------------------------------------------

\newcommand{\Cite}[1]{\,\cite{#1}}

% Make the captions have small ``Figure #'' 
\makeatletter
\renewcommand{\@makecaption}[2]{%
        \vskip\abovecaptionskip
        {\small #1:} {\small #2\par
        \vskip\belowcaptionskip}}
\makeatother

\newcommand{\figureTop}[1]{
  \begin{figure}[!t]{\sloppy #1}\end{figure}
}

\newcommand{\figureBot}[1]{
  \begin{figure}[!b]{\sloppy #1}\end{figure}
}


% Comment region command from Sara McMains
\newcommand{\comment}[1]{} 
% Make text red command (requires \usepackage[usenames]{color})
\newcommand{\textred}[1]{\textcolor{red}{#1}}
\newcommand {\note}[1]{{\color{magenta}\bf{Note: #1}\normalfont}}
\newcommand {\maneesh}[1]{{\color{red}\bf{MA: #1}\normalfont}}
\newcommand {\bjoern}[1]{{\color{blue}\bf{BH: #1}\normalfont}}

% get rid of comments
%\renewcommand {\maneesh}[1]{}
%\renewcommand {\bjoern}[1]{}


%--------------------------------------------------------------------

\begin{document}

\newlength{\origbaselineskip}
\setlength{\origbaselineskip}{\baselineskip}
%\setlength{\baselineskip}{.8\origbaselineskip}


%--------------------------------------------------------------------
\newcommand{\mytitle}{Biographical Sketch: Elena Leah Glassman}

%\pagestyle{plain}  %Plain style with page numbes
%\pagestyle{empty}  %Plain style with no page numbers
\pagestyle{fancy}
\lhead{\small\textsl{\mytitle}}
\rhead{}
\chead{}
\lfoot{}
\rfoot{}
\cfoot{\small\textsl{Page \thepage}} % Add in page numbers
%\cfoot{\small\textsl{}} % Remove page numbers
\renewcommand{\footrulewidth}{0pt}

\renewcommand{\thepage}{\arabic{page}}
\renewcommand{\thesection}{\hspace{-0.20in}}%{\Alph{section}} %bjoern
%\renewcommand{\thesubsection}{\arabic{subSection}}

\fancypagestyle{plain}{
  \fancyhf{}
  \cfoot{\small\textsl{Page \thepage}} 
  \renewcommand{\footrulewidth}{0pt}
  \renewcommand{\headrulewidth}{0pt}
}



%--------------------------------------------------------------------
%--------------------------------------------------------------------

\thispagestyle{plain}
\newlength{\spc}
\setlength{\spc}{.3in}

\newcommand{\Statement}[1]{#1\vspace{.15\spc}}
\newcommand{\Section   }[1]{\vspace{.15\spc}\textsc{\textbf{#1}}\vspace{.1\spc}}
\newcommand{\SubSection}[1]{\vspace{.10\spc}        \textbf{#1} \vspace{.1\spc}}

\newcommand{\PlainItem}[2]{\parbox[t]{.9in}{         }\hfill\parbox[t]{5.4in}{\textbf{#1}#2}\vspace{.15\spc}}
\newcommand{\DatedItem}[3]{\parbox[t]{.9in}{\small #1}\hfill\parbox[t]{5.4in}{\textbf{#2}#3}\vspace{.15\spc}}

\newcommand{\Pub}[1]{#1\vspace{.15\spc}}

\newcommand{\Junk}[1]{---\hfill\parbox[t]{431pt}{#1}\vspace{.15\spc}}

\newcommand{\StatementCat}[2]{\parbox[t]{1.1in}{\textbf{#1}}\hfill\parbox[t]{5.4in}{#2}\vspace{.15\spc}}

\begin{center}
\large
\textbf{Biographical Sketch: Elena Leah Glassman}\\
\vspace{0.05\spc}
\normalsize
Assistant Professor, Computer Science\\
SEAS \& the Radcliffe Institute for Advanced Study\\
Harvard University (Cambridge, MA, USA)\\
\small{\url{https://glassmanlab.seas.harvard.edu}}
\end{center}
%\vspace{0.05\spc}
%\hrule
%\vspace{-0.2in}

%\Section{Role in Project:}\\
%Director of Training Core; Investigator in User Needs Assessment Research %Project, Customization of Environments Development Project. 
%Responsible for User Needs Assessment, Development of Scenarios

\Section{Professional Preparation:}\\
Massachusetts Institute of Technology, Cambridge MA, \ BS in Electrical Sci \& Eng \hfill June '08\\
Massachusetts Institute of Technology, Cambridge MA, \ MEng in Electrical Eng \& Comp Sci \hfill Feb '10\\
Massachusetts Institute of Technology, Cambridge MA, \ PhD in Electrical Eng \& Comp Sci \hfill Sep '16\\
University of California, Berkeley, \ Postdoctoral Scholar, Electrical Eng \& Comp Sci \hfill '16-18
%\DatedItem{June 2008}{Massachusetts Institute of Technology (Cambridge, MA, USA)}{\\
%    Bachelor of Science in Electrical Science \& Engineering}
%\DatedItem{Feb 2010}{Massachusetts Institute of Technology (Cambridge, MA, USA)}{\\
%    Master of Engineering in Electrical Engineering \& Computer Science}
%\DatedItem{Sept 2016}{Massachusetts Institute of Technology (Cambridge, MA, USA)}{\\
%    Doctor of Philosophy in Electrical Engineering \& Computer Science\\
%    Thesis: \textit{Clustering and Visualizing Solution Variation in Massive Programming Classes}\\
%    Thesis Advisor: Rob Miller, Distinguished Professor of Computer Science }\\
%\DatedItem{2016--2018}{University of California, Berkeley (Berkeley, CA, USA)}{\\
%    Postdoctoral Scholar, EECS \& Berkeley Institute for Design\\
%    Fellow, Berkeley Institute of Data Science
%}


%----------------------------------------------------------------------------
\Section{Appointments:}\\
\DatedItem{2019--present}{Harvard University (Cambridge, MA, USA)}{  \\
            Assistant Professor, CS, SEAS \& Radcliffe Institute for Advanced Study
}
\DatedItem{2018--2018}{Harvard University (Cambridge, MA, USA)}{  \\
            Research Associate, CS, SEAS
}
\\

      %----------------------------------------------------------------------------

%\Section{Selected Awards}:\\
%      Sloan Research Fellowship (2013), NSF CAREER Award (2012), Okawa Foundation Research Award (2012), Qualcomm Faculty Fellow (2010--Present), CHI Best Paper Award (2007,2012), UIST Best Paper Award (2006,2008,2010) \\[2pt]

%------------------------------------------------------w----------------------

      
\Section{Products}\\
{\bf Five Most Relevant Products:}\\
% CHI 2018
1. ``How Proxy Tasks and Subjective Measures Can Be Misleading in Evaluating XAI Systems'' by Zana Bucinca*, Phoebe Lin*, Krzysztof Gajos, and Elena Glassman. {\em ACM Intelligent User Interfaces} (IUI) 2020. *indicates equal contribution\\[2pt]
2. ``Characterizing Developer Use of Automatically Generated Patches'' by Jos\'e Pablo Cambronero, Jiasi Shen, J\"urgen Cito, Elena Glassman, and Martin Rinard. IEEE VL/HCC 2019.\\[2pt]
3. ``Enabling Data-Driven API Design with Community Usage Data: A Need-Finding Study'' by Tianyi Zhang, Björn Hartmann, Miryung Kim, and Elena Glassman. To be published in \emph{ACM Conference on Human Factors in Computing Systems} (CHI) 2020.\\[2pt]
4. ``Visualizing API Usage Examples at Scale'' by Elena Glassman*, Tianyi Zhang*, Bj\"orn Hartmann, Miryung Kim. \emph{ACM Conference on Human Factors in Computing Systems} (CHI), Paper No. 580, 2018. *indicates equal contribution\\[2pt]
% L@S2017
5. ``Writing Reusable Code Feedback at Scale with Mixed-Initiative Program Synthesis'' by Andrew Head, Elena Glassman, Gustavo Soares, Ryo Suzuki, Lucas Figueredo, Loris D'Antoni, Bj\"orn Hartmann. {\em ACM Learning@Scale Conference}, pp.~89--98, 2017\\[2pt]



%----------------------------------------------------------------------------
%\clearpage
{\bf Five Other Products:}\\
% CHI2018
1. ``Interactive Extraction of Examples from Existing Code'' by Andrew Head, Elena Glassman, Bj\"orn Hartmann, Marti Hearst. \emph{ACM Conference on Human Factors in Computing Systems} (CHI), Paper No. 85, 2018. {\em (Honorable Mention---Top 4\%)}\\[2pt]
% CSCW16
2. ``Learnersourcing Personalized Hints'' by Elena Glassman, Aaron Lin, Carrie Cai, Rob Miller. \emph{ACM Computer-Supported Cooperative Work and Social Computing} (CSCW), pp.~1626--1636, 2016.\\[2pt]
% TOCHI 2015
3. ``OverCode: visualizing variation in student solutions to programming problems at scale'' by Elena Glassman, Jeremy Scott, Rishabh Singh, Philip Guo, Robert Miller. \emph{ACM Transactions on Computer-Human Interaction} ({\small Special Issue on Online Learning at Scale}), 22 (2), April 2015.\\[2pt]
% Foobaz
4. ``Foobaz: Variable Name Feedback for Student Code at Scale'' by Elena Glassman, Lyla Fischer, Jeremy Scott, Robert Miller. \emph{ACM Symposium on User Interface Software \& Technology} (UIST), pp.~609--617, 2015.\\[2pt]
% DocMatrix
%3. ``DocMatrix: Self-Teaching from Multiple Sources'' by Elena Glassman, Daniel Russell. \emph{Proceedings of the Association for Information Science and Technology}, 53 (1), 2016.\\[2pt]
5. ``Mudslide: A Spatially Anchored Census of Student Confusion for Online Lecture Videos'' by Elena Glassman, Juho Kim, Andr\'es Monroy-Hern\'andez, Merrie Ringel Morris. \emph{ACM Conference on Human Factors in Computing Systems} (CHI), pp.~1555--1564, 2015. {\em (Honorable Mention---Top 5\%)}\\[2pt]
%5. ``RIMES: Embedding Interactive Multimedia Exercises in Lecture Videos'' by Juho Kim, Elena Glassman, Andr\'es Monroy-Hern\'andez, Merrie Ringel Morris. \emph{ACM Conference on Human Factors in Computing Systems} (CHI), pp.~1535--1544, 2015.\\[2pt]



%----------------------------------------------------------------------------
\Section{Synergistic Activities:}\\
%innovations in teaching and training (e.g., development of curricular materials and pedagogical methods); contributions to the science of learning; development and/or refinement of research tools; computation methodologies and algorithms for problem-solving; development of databases to support research and education; broadening the participation of groups underrepresented in STEM; and service to the scientific and engineering community outside of the individual’s immediate organization.
1. Awarded the MIT Amar Bose Teaching Fellowship, for developing and deploying innovative tools for teaching computer science at scale at MIT, such as a tool for eliciting and distributing personalized learnersourced debugging and design optimization hints [Glassman et al. 2016] in MIT's undergraduate digital architecture class (approx. 200 students per semester).\\[2pt]
2. Developed and taught introductory computer science curriculum for, and later served on the leadership team for, the Middle East Education through Technology organization, which brings together high-school-age Israeli and Palestinian boys and girls to create positive change through technology and entrepreneurship, in partnership with the Massachusetts Institute of Technology (MIT).\\[2pt]
%3. Participated in the MIT Teaching \& Learning Lab Graduate Teaching ...\\
3. Worked with University of California, Berkeley's CS61A course staff to deploy \ttt{OverCode} [Glassman et al. 2015] to help graders compose and deliver composition feedback and grades to {\bf over 1500 students} in a single semester.\\[2pt]
4. Co-led the 2019 ACM User Interface Software \& Technology Doctoral Symposium to coach a diverse group of promising graduating PhD students in human-computer interaction reach their goals of graduating with a strong thesis and finding a research job that fits their needs and desires.\\[2pt]
5. Co-organized the 2019 PLATEAU workshop that brings together researchers in programming languages and human-computer interaction.\\[2pt]

%5. Led an MIT M.Eng. student through the process of deploying a variant of the same tool, \ttt{GroverCode}, to fulfill her thesis requirement and support the composition grading of hundreds of MIT student-written programs submitted in the MIT EECS department's introductory programming course.\\[2pt]
%5. Co-taught the combined undergraduate/graduate user interface design course that was part of the core curriculum for CS students in the MIT EECS department (approx. 175 students).
%1. Faculty Director for the Jacobs Institute for Design Innovation at UC Berkeley, a new undergraduate engineering design institute at home in the newly constructed Jacobs Hall. (CTO Spring 2014--2016, Director 2016--Present, building opened in Fall 2015.)\\{\small \url{http://jacobsinstitute.berkeley.edu}}\\[2pt]
%2. Co-Founder and Co-Director of the CITRIS Invention Lab, a rapid prototyping facility and experiential learning studio that focuses on wireless electronics and digital fabrication tools. (2012--Present)\\{\small \url{http://inventionlab.org}} \\[2pt]
%2. Participated in Presidential Chair Fellows teaching improvement program at UC Berkeley. Received the CS Division’s Jim and Donna Gray Faculty Award for Excellence in Undergraduate Teaching and the Diane S. McEntyre Award for Excellence in Teaching.\\[2pt]
%3. Advisor for underrepresented undergraduate students participating in NSF REU summer research programs at UC Berkeley (2010–present).

%4. Developed new class on interactive device design at UC Berkeley, taught annually since Fall 2012.\\{\small \url{http://hci.berkeley.edu/devicedesign}}\\[2pt]

%4. Co-sponsored post-doctoral position for M. Etemadi, split between UCB and UCSF (2014).
%2. Re-developed undergraduate class on user interface design at UC Berkeley.\\[2pt]
%3. Initiated undergraduate certificate program {\em Course Thread on Human-Centered Design}, administered by the Berkeley Townsend Center for the Humanities.\\[2pt]
%3. Organized workshop series on user interface design techniques.\\[2pt]
%3. Participated in Presidential Chair Fellows teaching improvement program at UC Berkeley.\\[2pt]
%4. Re-developed graduate class on HCI research at UC Berkeley.\\[2pt]
%5. Advisor to MobileWorks, a UC Berkeley crowdsourcing startup company. (2011--Present)
%1. Developed graduate class on crowdsourcing at UC Berkeley. (Spring 2011)\\[2pt]
%2. Co-organized workshop on crowdsourcing at CHI 2011 in Vancouver, Canada. The workshop had more than 50 attendees.\\[2pt]
%4. Organized and taught two-day course on interactive graphics programming for art and design students at California College of Arts. (Summer 2009)

%\Section{Collaborators (33 collaborators at 17 institutions):}\\
%Alice Agogino (Berkeley), Maneesh Agrawala (Berkeley), Rajeev Alur (UPenn), Michel Beaudouin-Lafon (Univ. Paris), Michael Bernstein (Stanford), Ras Bodik (Berkeley), Jan Borchers (RWTH Aachen), Tony DeRose (Pixar), Mira Dontcheva (Adobe), Steven Dow (CMU), Mozziyar Etemadi (UCSF), George Fitzmaurice (Autodesk), Sean Follmer (MIT Media Lab), Armando Fox (Berkeley), Tovi Grossman (Autodesk), Sumit Gulwani (MSR), Marti Hearst (Berkeley), Scott Klemmer (PhD advisor - UCSD), Edward Lee (Berkeley), Philip Levis (Stanford), Wilmot Li (Adobe), Kurt Luther (Virginia Tech), Wendy Mackay (INRIA), Lena Mamykina (Columbia), Lora Oehlberg (INRIA), Eric Paulos (Berkeley), Tapan Parikh (Berkeley), Shuvo Roy (UCSF), Ryan Schmidt (Autodesk), Koushik Sen (Berkeley), Mahesh Viswanathan (UIUC), Li-Yi Wei (MSR), Paul Wright (Berkeley).\\[2pt]
%Joel Brandt (Adobe), 
%----------------------------------------------------------------------------
%\vspace{-0.1in}
%\Section{Graduate Advisors:}\\
%Scott R. Klemmer (Ph.D., Stanford), Norm Badler (MSE, University of Pennsylvania)\\[2pt]
%\vspace{-0.1in}

%\Section{Thesis Advisor and Postgraduate-Scholar Sponsor (9 current, at 3 institutions):}\\
%Postdoc (2): Gustavo Soares, David Mellis (Berkeley)\\
%Ph.D. (6): Peggy Chi, Valkyrie Savage, Amy Pavel, Andrew Head (all Berkeley), Julie Newcomb (UW), Will McGrath (Stanford)\\
%MS (1): Michelle Nguyen (Berkeley)
%MS (3): Wei Wu (UC Berkeley), Philipp Gutheim (UC Berkeley), Drew Fisher (UC Berkeley), Alvin Yuan (Berkeley); \\
%Postgraduate (1): Mozziyar Etemadi (UC Berkeley/UC San Francisco)\\[2pt]
%----------------------------------------------------------------------------
%--------------------------------------------------------------------
%--------------------------------------------------------------------
\end{document}
